 
\documentclass[12pt]{article}
%%% Работа с русским языком

\usepackage{cmap}					% поиск в PDF
\usepackage{mathtext} 				% русские буквы в фомулах
\usepackage[T2A]{fontenc}			% кодировка
\usepackage[utf8]{inputenc}			% кодировка исходного текста
\usepackage[english,russian]{babel}	% локализация и переносы

%%% Дополнительная работа с математикой
\usepackage{amsmath,amsfonts,amssymb,amsthm,mathtools} % AMS
\usepackage{icomma} % "Умная" запятая: $0,2$ --- число, $0, 2$ --- перечисление

%% Номера формул
\mathtoolsset{showonlyrefs=true} % Показывать номера только у тех формул, на которые есть \eqref{} в тексте.

%% Шрифты
\usepackage{euscript}	 % Шрифт Евклид
\usepackage{mathrsfs} % Красивый матшрифт

%Листинг кода
\usepackage{listings}
\usepackage{color}
\usepackage{graphicx}
\graphicspath{}
\DeclareGraphicsExtensions{.pdf,.png,.jpg}
%Задание параметров страницы
% в преамбуле
\usepackage{caption} 
\usepackage[left=1.5cm,right=1cm,
top=1cm,bottom=1cm,bindingoffset=0cm]{geometry}


\usepackage[T2A]{fontenc}
\usepackage{indentfirst}
\usepackage{listings}
\usepackage{color}
\usepackage{misccorr}
\usepackage{graphicx}
\usepackage{cancel}
\graphicspath{ {images/} }
\usepackage{amsmath}
\usepackage{ulem}
\textheight=24cm 
\textwidth=16cm 
\oddsidemargin=0pt
\topmargin=-1.5cm 
\parindent=24pt 
\parskip=0pt 
\tolerance=2000 
\flushbottom     


\begin{document}
	
	\begin{titlepage}
		\begin{center}
			\large
			МИНИСТЕРСТВО НАУКИ И ВЫСШЕГО ОБРАЗОВАНИЯ\\ РОССИЙСКОЙ ФЕДЕРАЦИИ \\
			\vspace{0.05cm}
			ФЕДЕРАЛЬНОЕ ГОСУДАРСТВЕННОЕ БЮДЖЕТНОЕ\\ ОБРАЗОВАТЕЛЬНОЕ УЧРЕЖДЕНИЕ \\ ВЫСШЕГО ОБРАЗОВАНИЯ 
			
			\vspace{0.25cm}
			
			НОВОСИБИРСКИЙ ГОСУДАРСТВЕННЫЙ ТЕХНИЧЕСКИЙ УНИВЕРСИТЕТ
			\vspace{0.5cm}
			
			
			Кафедра Автоматизированных Систем Управления
			\vfill
			
			
			\textbf{ОТЧЁТ ПО ЛАБОРАТОРНОЙ РАБОТЕ №1}\\[2mm]
			\textbf{по основам теории управления}\\[2mm]
			\textbf{Математическое моделирование динамических систем}\\[3mm]
			\vspace{0.1cm}
		\end{center}
		\vfill
		\newlength{\ML}
		\settowidth{\ML}{«\underline{\hspace{0.7cm}}» \underline{\hspace{2cm}}}
		\hfill\begin{minipage}{0.3\textwidth}
			Выполнил:\\
			Студент гр. АВТ-813, АВТФ \\
			
			{\it Чернаков} \\
			{\it Кирилл Олегович}\\
			
		\end{minipage}%
		\bigskip
		
		\hfill\begin{minipage}{0.3\textwidth}
			Проверил:\\
			к.т.н., Доцент, заведующий каф. АСУ \\
			{\it Достовалов} \\
			{\it Дмитрий Николаевич}
		\end{minipage}%
		\vfill
		
		\begin{center}
			Новосибирск, 2020 г.
		\end{center}
	\end{titlepage}  
	\tableofcontents
	\newpage
	
	\definecolor{mygreen}{rgb}{0,0.6,0}
	\definecolor{mygray}{rgb}{0.5,0.5,0.5}
	\definecolor{mymauve}{rgb}{0.63,0.082,0.082}
	
	\lstset{ %
		backgroundcolor=\color{white},   % choose the background color; you must add \usepackage{color} or \usepackage{xcolor}
		basicstyle=\footnotesize,        % the size of the fonts that are used for the code
		breakatwhitespace=false,         % sets if automatic breaks should only happen at whitespace
		breaklines=true,                 % sets automatic line breaking
		captionpos=b,                    % sets the caption-position to bottom
		commentstyle=\color{mygreen},    % comment style
		deletekeywords={...},            % if you want to delete keywords from the given language
		escapeinside={\%*}{*)},          % if you want to add LaTeX within your code
		extendedchars=\true,              % lets you use non-ASCII characters; for 8-bits encodings only, does not work with UTF-8
		frame=false,                    % adds a frame around the code
		keepspaces=true,                 % keeps spaces in text, useful for keeping indentation of code (possibly needs columns=flexible)
		keywordstyle=\color{blue},       % keyword style
		language=C++,                 % the language of the code
		morekeywords={*,...},            % if you want to add more keywords to the set
		numbers=left,                    % where to put the line-numbers; possible values are (none, left, right)
		numbersep=5pt,                   % how far the line-numbers are from the code
		numberstyle=\tiny\color{black}, % the style that is used for the line-numbers
		rulecolor=\color{white},         % if not set, the frame-color may be changed on line-breaks within not-black text (e.g. comments (green here))
		showspaces=false,                % show spaces everywhere adding particular underscores; it overrides 'showstringspaces'
		showstringspaces=false,          % underline spaces within strings only
		showtabs=false,                  % show tabs within strings adding particular underscores
		stepnumber=1,                    % the step between two line-numbers. If it's 1, each line will be numbered
		stringstyle=\color{mymauve},     % string literal style
		tabsize=4,                       % sets default tabsize to 2 spaces
		title=\lstname                   % show the filename of files included with \lstinputlisting; also try caption instead of title
	}
	
	\section{Постановка задачи}
	\begin{enumerate} 
		\item Локализовать корни с помощью  построения графика (Desmos)	
		\item Разработать программную реализацию трех методов уточнения корней:
		\begin{enumerate} 
			\item Ньютона,
			\item простых итераций,
			\item метода заданного в таблице вариантов.
		\end{enumerate}
		\item Произвести вычисления с различной точностью и сравнить количество итераций для нахождения корней различными методами.	
	\end{enumerate}
	\newpage  
	
	\section{Аналитическое решение задачи}

		\begin{equation}
			\begin{split}
			
				x\prime \prime +x\prime =t, x(0)=x\prime (0)=0 \\
				x\rightarrow X(p) \\
				x\prime\rightarrow pX(p) \\
				x\prime\prime\rightarrow p^{2}X(p) \\
				t\rightarrow\frac{1}{p^{2}} \\
				p^{2}X(p)+pX(p)=\frac{1}{p^{2}} \\
				p^{4}X(p)+p^{3}X(p)=1 \\
				X(p)=\frac{1}{p^{4}+p^{3}}=\frac{1}{p^{3}(p+1)} \\
			\end{split}
		\end{equation}

	\newpage 
	
	\section{Структурные схемы в Matlab}
	
	\newpage 
	
	\section{Полученные графики}
	\begin{figure}[h]
%		\center{\includegraphics[scale=0.5]{1.png}}
		\caption{Контрольный пример}
	\end{figure}
	\newpage 
	
	\section{Выводы по задаче}
	
	\newpage 
	
	\section{Структурные схемы для примера 2}
	
	\newpage 
	
	\section{Графики}
	
	\newpage 
	
	\section{Заключение}
	Исходя из расчётов программы можно сделать вывод о том, что меньше всего итераций занял метод Ньютона за ним следует метод половинного деления и в конце метод простых итераций. 
	\newpage
	
	\section{Начальные приближения}
	Исходя из построенного графика найдём необходимые отрезки для поиска корней уравнения. Всего в уравнении 3 корня это можно узнать по пересечению графика c осью абсцисс в точках:
	\begin{enumerate} 
		\item $x1 = 2.843$
		\item $x2 = 0.693$
		\item $x3 = -2.537$
	\end{enumerate}
	Рассмотрим интервалы для каждого корня (брать интервалы будем, опираясь на изменение знака функции на концах отрезка и на наличие единственного корня в этом отрезке):
	\begin{enumerate} 
		\item $[2 ; 10000]$
		\item $[0 ;1]$
		\item $[-3 ; -2]$
	\end{enumerate} 
	\newpage
	
	
	
\end{document}