 
\documentclass[12pt]{article}
%%% Работа с русским языком

\usepackage{cmap}					% поиск в PDF
\usepackage{mathtext} 				% русские буквы в фомулах
\usepackage[T2A]{fontenc}			% кодировка
\usepackage[utf8]{inputenc}			% кодировка исходного текста
\usepackage[english,russian]{babel}	% локализация и переносы

%%% Дополнительная работа с математикой
\usepackage{amsmath,amsfonts,amssymb,amsthm,mathtools} % AMS
\usepackage{icomma} % "Умная" запятая: $0,2$ --- число, $0, 2$ --- перечисление

%% Номера формул
\mathtoolsset{showonlyrefs=true} % Показывать номера только у тех формул, на которые есть \eqref{} в тексте.

%% Шрифты
\usepackage{euscript}	 % Шрифт Евклид
\usepackage{mathrsfs} % Красивый матшрифт

%Листинг кода
\usepackage{listings}
\usepackage{color}
\usepackage{graphicx}
\graphicspath{}
\DeclareGraphicsExtensions{.pdf,.png,.jpg}
%Задание параметров страницы
% в преамбуле
\usepackage{caption} 
\usepackage[left=1.5cm,right=1cm,
top=1cm,bottom=1cm,bindingoffset=0cm]{geometry}


\usepackage[T2A]{fontenc}
\usepackage{indentfirst}
\usepackage{listings}
\usepackage{color}
\usepackage{misccorr}
\usepackage{graphicx}
\usepackage{cancel}
\graphicspath{ {images/} }
\usepackage{amsmath}
\usepackage{ulem}
\textheight=24cm 
\textwidth=16cm 
\oddsidemargin=0pt
\topmargin=-1.5cm 
\parindent=24pt 
\parskip=0pt 
\tolerance=2000 
\flushbottom     


\begin{document}
	
	\begin{titlepage}
		\begin{center}
			\large
			МИНИСТЕРСТВО НАУКИ И ВЫСШЕГО ОБРАЗОВАНИЯ\\ РОССИЙСКОЙ ФЕДЕРАЦИИ \\
			\vspace{0.05cm}
			ФЕДЕРАЛЬНОЕ ГОСУДАРСТВЕННОЕ БЮДЖЕТНОЕ\\ ОБРАЗОВАТЕЛЬНОЕ УЧРЕЖДЕНИЕ \\ ВЫСШЕГО ОБРАЗОВАНИЯ 
			
			\vspace{0.25cm}
			
			НОВОСИБИРСКИЙ ГОСУДАРСТВЕННЫЙ ТЕХНИЧЕСКИЙ УНИВЕРСИТЕТ
			\vspace{0.5cm}
			
			
			Кафедра Автоматизированных Систем Управления
			\vfill
			
			
			\textbf{ОТЧЁТ ПО ЛАБОРАТОРНОЙ РАБОТЕ №1}\\[2mm]
			\textbf{по основам теории управления}\\[2mm]
			\textbf{Математическое моделирование динамических систем}\\[3mm]
			\vspace{0.1cm}
		\end{center}
		\vfill
		\newlength{\ML}
		\settowidth{\ML}{«\underline{\hspace{0.7cm}}» \underline{\hspace{2cm}}}
		\hfill\begin{minipage}{0.3\textwidth}
			Выполнил:\\
			Студент гр. АВТ-813, АВТФ \\
			
			{\it Кинчаров} \\
			{\it Данил Дмитриевич}\\
			{\it Пайхаев} \\
			{\it Алексей Евгеньевич} \\
			{\it Чернаков} \\
			{\it Кирилл Олегович}\\
			
		\end{minipage}%
		\bigskip
		
		\hfill\begin{minipage}{0.3\textwidth}
			Проверил:\\
			к.т.н., Доцент, заведующий каф. АСУ \\
			{\it Достовалов} \\
			{\it Дмитрий Николаевич}
		\end{minipage}%
		\vfill
		
		\begin{center}
			Новосибирск, 2020 г.
		\end{center}
	\end{titlepage}  
	\tableofcontents
	\newpage
	
	\definecolor{mygreen}{rgb}{0,0.6,0}
	\definecolor{mygray}{rgb}{0.5,0.5,0.5}
	\definecolor{mymauve}{rgb}{0.63,0.082,0.082}
	
	\lstset{ %
		backgroundcolor=\color{white},   % choose the background color; you must add \usepackage{color} or \usepackage{xcolor}
		basicstyle=\footnotesize,        % the size of the fonts that are used for the code
		breakatwhitespace=false,         % sets if automatic breaks should only happen at whitespace
		breaklines=true,                 % sets automatic line breaking
		captionpos=b,                    % sets the caption-position to bottom
		commentstyle=\color{mygreen},    % comment style
		deletekeywords={...},            % if you want to delete keywords from the given language
		escapeinside={\%*}{*)},          % if you want to add LaTeX within your code
		extendedchars=\true,              % lets you use non-ASCII characters; for 8-bits encodings only, does not work with UTF-8
		frame=false,                    % adds a frame around the code
		keepspaces=true,                 % keeps spaces in text, useful for keeping indentation of code (possibly needs columns=flexible)
		keywordstyle=\color{blue},       % keyword style
		language=C++,                 % the language of the code
		morekeywords={*,...},            % if you want to add more keywords to the set
		numbers=left,                    % where to put the line-numbers; possible values are (none, left, right)
		numbersep=5pt,                   % how far the line-numbers are from the code
		numberstyle=\tiny\color{black}, % the style that is used for the line-numbers
		rulecolor=\color{white},         % if not set, the frame-color may be changed on line-breaks within not-black text (e.g. comments (green here))
		showspaces=false,                % show spaces everywhere adding particular underscores; it overrides 'showstringspaces'
		showstringspaces=false,          % underline spaces within strings only
		showtabs=false,                  % show tabs within strings adding particular underscores
		stepnumber=1,                    % the step between two line-numbers. If it's 1, each line will be numbered
		stringstyle=\color{mymauve},     % string literal style
		tabsize=4,                       % sets default tabsize to 2 spaces
		title=\lstname                   % show the filename of files included with \lstinputlisting; also try caption instead of title
	}

	\section{Цель работы}
	
	\section{Аналитическое решение задачи}
	\subsection{Вариант - 19}	
	$
	\\
		x’ ’ +x’ =t, x(0)=x’ (0)=0 \\
		x\rightarrow X(p) \\
		x’\rightarrow pX(p) \\
		x’’\rightarrow p^{2}X(p) \\
		t\rightarrow\frac{1}{p^{2}} \\
		p^{2}X(p)+pX(p)=\frac{1}{p^{2}} \\
		p^{4}X(p)+p^{3}X(p)=1 \\
		X(p)=\frac{1}{p^{4}+p^{3}}=\frac{1}{p^{3}(p+1)} \\
		X(p)=\frac{p^{2}-p+1}{p^{3}}-\frac{1}{p+1}=1-t+\frac{t^{2}}{2}-exp(-t) \\
	$
	\newpage 
	 
			\subsection{Вариант - 8}	
		\begin{figure}[h]
		\center{\includegraphics[scale=0.42]{8/1.jpg}}
	%	\caption{Структурная схема -- ДУ относительно старшей производной $x’’=t-x’$}
	\end{figure}	
		\clearpage
				\subsection{Вариант - 15}	
	\begin{figure}[h]
		\center{\includegraphics[scale=0.15]{15/6.jpg}}
		%	\caption{Структурная схема -- ДУ относительно старшей производной $x’’=t-x’$}
	\end{figure}	
			\clearpage
	\section{Структурные схемы в Matlab}
	
	\begin{figure}[h]
		\center{\includegraphics[scale=0.42]{4.png}}
		\caption{Структурная схема -- ДУ относительно старшей производной $x’’=t-x’$}
	\end{figure}

	\begin{figure}[h]
	\center{\includegraphics[scale=0.5]{15/4.png}}
	\caption{Структурная схема 15 -- ДУ относительно старшей производной}
\end{figure}

\begin{figure}[h]
	\center{\includegraphics[scale=0.5]{8/5.png}}
	\caption{Структурная схема 8 -- ДУ относительно старшей производной}
\end{figure}


	\begin{figure}[h]
		\center{\includegraphics[scale=0.45]{5.png}}
		\caption{Структурная схема -- уравнение движения $x(t)=1-t+\frac{t^{2}}{2}-exp(-t)$}
	\end{figure}

	\begin{figure}[h]
	\center{\includegraphics[scale=0.5]{15/5.png}}
	\caption{Структурная схема 15 -- уравнение движения}
\end{figure}
	
	\begin{figure}[h]
	\center{\includegraphics[scale=0.5]{8/6.png}}
	\caption{Структурная схема 8 -- уравнение движения}
\end{figure}	
	
	\newpage 
	\clearpage
	\section{Полученные графики}
	\begin{figure}[h]
		\center{\includegraphics[scale=0.25]{1.png}}
		\caption{Полученный результат -- ДУ относительно старшей производной $x’’=t-x’$}
	\end{figure}

	\begin{figure}[h]
		\center{\includegraphics[scale=0.25]{2.png}}
		\caption{Полученный результат --  уравнение движения $x(t)=1-t+\frac{t^{2}}{2}-exp(-t)$}
	\end{figure}

	\begin{figure}[h]
	\center{\includegraphics[scale=0.25]{3.png}}
	\caption{Полученный результат --  19 сравнение уравнения движения и ДУ относительно старшей производной}
\end{figure}

	\begin{figure}[h]
	\center{\includegraphics[scale=0.4]{8/2.png}}
	\caption{Полученный результат -- 8 ДУ относительно старшей производной}
\end{figure}

\begin{figure}[h]
	\center{\includegraphics[scale=0.4]{8/4.png}}
	\caption{Полученный результат --  8 уравнение движения}
\end{figure}

	\begin{figure}[h]
	\center{\includegraphics[scale=0.4]{8/3.png}}
	\caption{Полученный результат --  8 сравнение уравнения движения и ДУ относительно старшей производной}
\end{figure}

	\begin{figure}[h]
	\center{\includegraphics[scale=0.3]{15/1.png}}
	\caption{Полученный результат -- 15 ДУ относительно старшей производной}
\end{figure}

\begin{figure}[h]
	\center{\includegraphics[scale=0.3]{15/2.png}}
	\caption{Полученный результат --  15 уравнение движения}
\end{figure}

\begin{figure}[h]
	\center{\includegraphics[scale=0.3]{15/3.png}}
	\caption{Полученный результат --  15 сравнение уравнения движения и ДУ относительно старшей производной}
\end{figure}

\begin{figure}[h]
	\center{\includegraphics[scale=0.45]{6.png}}
	\caption{Полученный результат --  8 график относительной погрешности}
\end{figure}


	\newpage 
		\clearpage
	\section{Выводы по задаче}
	В ходе работы была построена и исследована динамическая система двумя способами: аналитическим решением диф. уравнений и моделированием в Matlab. Сравним полученные результаты с помощью отношения аналитического решениея к численному решению, получим уравнение, график которого близок к единице. Сравним полученные результаты с помощью разности аналитического решениея и численного, получим уравнение, график которого близок к нулю. На основании полученных результатов можно говорить о достоверности решения. 
	\newpage 
		\clearpage
	\section{Пример 2}
	\subsection{Структурная схема}
	\begin{figure}[h]
		\center{\includegraphics[scale=0.6]{структурная схема для 2 примера.png}}
		%\caption{Полученный результат --  8 график относительной погрешности}
	\end{figure}
	\clearpage
 	\subsection{Графики}
 	\begin{figure}[h]
 		\center{\includegraphics[scale=1]{теорреш1.png}}
 		\caption{Полученный результат -- x1 теоретическое решение}
 	\end{figure}
 \begin{figure}[h]
 	\center{\includegraphics[scale=1]{теорреш2.png}}
 	\caption{Полученный результат -- x2 теоретическое решение}
 \end{figure}

\begin{figure}[h]
	\center{\includegraphics[scale=1]{аналитрешх1.png}}
	\caption{Полученный результат -- x1 аналитическое решение}
\end{figure}
\begin{figure}[h]
	\center{\includegraphics[scale=1]{аналитрешх2.png}}
	\caption{Полученный результат -- x2 аналитическое решение}
\end{figure}

\begin{figure}[h]
	\center{\includegraphics[scale=0.5]{относпогрешх1.png}}
	\caption{Полученный результат -- x1 график относительной погрешности решение}
\end{figure}
\begin{figure}[h]
	\center{\includegraphics[scale=0.4]{относпогрешх2.png}}
	\caption{Полученный результат -- x2 график относительной погрешности решение}
\end{figure}
	\clearpage
	 \subsection{Вывод}
	 	В ходе работы была исследована система ДУ с помощью двух способов: аналитическим решением дифференциальных уравнений и методом моделирования в Matlab. В результате были получены графики относительной погрешности, которая находится в пределах нуля. Отклонение от нуля на обоих графиках составляет примерно ${10^-}^6$, что является незначительным.
	\clearpage
	\section{Дополнительное задание}
	
	\subsection{Аналитическое решение}
	
	\begin{figure}[h]
		\center{\includegraphics[scale=0.2]{7.jpg}}
		%\caption{Полученный результат --  8 график относительной погрешности}
	\end{figure}
	\begin{figure}[h]
	\center{\includegraphics[scale=0.15]{8.jpg}}
	%\caption{Полученный результат --  8 график относительной погрешности}
\end{figure}
	\begin{figure}[h]
	\center{\includegraphics[scale=0.15]{9.jpg}}
	%\caption{Полученный результат --  8 график относительной погрешности}
\end{figure}
	\newpage 
		\clearpage
	\subsection{Структурная схема}
\begin{figure}[h]
	\center{\includegraphics[scale=0.5]{7.png}}
	%\caption{Полученный результат --  8 график относительной погрешности}
\end{figure}
		\clearpage
	\subsection{Полученные графики -- x1}
	
\begin{figure}[h]
	\center{\includegraphics[scale=0.5]{решение 1.png}}
	\caption{Полученный результат -- x1 теоретическое решение}
\end{figure}

\begin{figure}[h]
	\center{\includegraphics[scale=0.55]{решение 1.png}}
	\caption{Полученный результат -- x1 аналитическое решение}
\end{figure}

\begin{figure}[h]
	\center{\includegraphics[scale=0.4]{относительное первое решение.png}}
	\caption{Полученный результат -- x1 график относительной погрешности решение}
\end{figure}
		\clearpage
		\subsection{Полученные графики -- x2}
		
\begin{figure}[h]
	\center{\includegraphics[scale=0.5]{решение 2.png}}
	\caption{Полученный результат -- x2 теоретическое решение}
\end{figure}

\begin{figure}[h]
	\center{\includegraphics[scale=0.5]{решение 2.png}}
	\caption{Полученный результат -- x2 аналитическое решение}
\end{figure}

\begin{figure}[h]
	\center{\includegraphics[scale=0.35]{относительное решение 2.png}}
	\caption{Полученный результат -- x2 график относительной погрешности решение}
\end{figure}
	\clearpage
		\subsection{Выводы по задаче}
		
	В ходе работы была исследована система ДУ с помощью двух способов: аналитическим решением дифференциальных уравнений и методом моделирования в Matlab. В результате были получены графики относительной погрешности, которая находится в пределах нуля. Отклонение от нуля на обоих графиках составляет примерно ${10^-}^4$, что является незначительным.	
	
	\newpage 
		
\end{document}